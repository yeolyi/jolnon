\documentclass[ko]{snu-cse-bsc-thesis}

% Add your packages here, e.g.,
% \usepackage{tikz}
\usepackage{siunitx}

% For lorem ipsum; remove these lines when writing your thesis
\usepackage{lipsum}
\usepackage{jiwonlipsum}

% hyperref *must* be the last package to be loaded!
\usepackage[pdfusetitle]{hyperref}

\addbibresource{bib.bib}

\title{키보드 접근성 강화를 통한 학부 홈페이지의 UX 개선}
\titlealt{Improving UX of Undergraduate Department Websites through Enhanced Keyboard Accessibility}
\author{이성열}
\advisor{서진욱}
\date{2024년 12월 13일}
\approvaldate{2024년 12월}

\koreankeywords{웹 접근성, 웹 접근성 평가, 장애인, WCAG, 대학 홈페이지}
\englishkeywords{Web Accessibility, Web Accessibility Testing, Disabled People, WCAG, University website}


\begin{document}
\maketitle

\pagenumbering{roman}
\begin{abstract}
다양한 인터렉션이 적극적으로 사용됨에 따라 모던 웹사이트는 점점 복잡해지고 있다. 이로인해 접근성을 준수한 웹 개발은 까다로워지고 있지만 접근성 준수의 필요성은 점점 강조되고있다. 컴퓨터공학부 홈페이지는 외부에서 학부와 연결되는 공적인 수단이기에 접근성에 대한 고려가 중요하다. 따라서 본 연구에서는 학부 홈페이지의 웹 접근성을 개선하고자한다. 다양한 장애 환경중에서도 포인팅 장치를 사용할 수 없는 운동장애 사용자들을 타겟으로 키보드만으로도 모든 정보의 접근 및 조작이 가능하도록 개선하고자한다. 이러한 개선 사항을 기록함으로서 추후 시각장애나 청각장애 사용자의 접근성 개선을 위한 자료로서 활용할 수 있을 것이다. 
\end{abstract}

\tableofcontents
\listoftables
\listoffigures

\chapter{서론}\label{chap:introduction}
\pagenumbering{arabic}
본 템플릿의 구성은 다음과 같다.
\ref{chap:body}장 본론의 \ref{sec:picture}절에서 그림의 예시를 보여준다.
\ref{sec:table}절에서 표의 예시를 보여준다.
\ref{chap:conclusion}장에서는 본 템플릿을 요약한다.

\section{절 예시}\label{sec:section}
\jiwon[2-3]


\chapter{본론}\label{chap:body}
정보 엔트로피는 각 메시지에 포함된 정보의 기댓값으로 식~\eqref{eq:entropy}\와 같다~\cite{6773024}.
\begin{equation}\label{eq:entropy}
  H(X) = -\sum_{i=1}^n {\mathrm{P}(x_i) \log_b \mathrm{P}(x_i)}
\end{equation}

\jiwon[4-6]


\section{그림}\label{sec:picture}
그림 예시는 그림~\ref{fig:example}\와 같다. 그림~\ref{fig:snu}\은 서울대학교 로고이고 그림~\ref{fig:eng}\는 서울대학교 공과대학 로고이다.

\begin{figure}[htp]
  \centering
  \begin{subfigure}[b]{0.5\textwidth}
    \centering
    \includegraphics[width=0.5\textwidth]{logo1.pdf}
    \bicaption{서울대학교 로고}{The logo of Seoul National University}\label{fig:snu}
  \end{subfigure}%
  \begin{subfigure}[b]{0.5\textwidth}
    \centering
    \includegraphics[width=0.9\textwidth]{logo2.pdf}
    \bicaption{공과대학 로고}{The logo of College of Engineering}\label{fig:eng}
  \end{subfigure}
  \bicaption[그림 예시 (목차 항목)]{그림 예시.}{An example of a figure.}\label{fig:example}
\end{figure}

\jiwon[7-8]


\section{표}\label{sec:table}
표 예시는 표~\ref{tab:example}\과 같다.\footnote{\jiwon[12]}

\begin{table}[htp]
  \centering
  \bicaption[표 예시 (목차 항목)]{표 예시.}{An example of a table.}\label{tab:example}
  \begin{tblr}{cc}
    \toprule
    상수 & 값 \\\midrule
    $c$ & \SI{299792458}{\meter\per\second} \\
    $h$ & \SI{6.62607015e-34}{\joule\per\hertz} \\\bottomrule
  \end{tblr}
\end{table}

\jiwon[9-10]


\chapter{결론}\label{chap:conclusion}
\jiwon[11]

\printbibliography

\begin{abstract}[en]
  영문 초록 작성 예정
\end{abstract}
\end{document}
